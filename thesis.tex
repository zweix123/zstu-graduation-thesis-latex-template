% !TeX spellcheck = en_US
% 特殊注释, 告知编译器使用美国英语作为拼写检查的语言

\documentclass[UTF8,a4paper,twoside]{ctexrep}  % 选项分别是UTF-8编码, A4纸张大小, 双面打印
% \documentclass[a4paper,UTF8,fontset=none]{ctexrep}  % fontset=none是为了后面指定自定义字体

\usepackage{amsmath,amssymb,amsfonts}  % 数学相关
\usepackage{ctex}
\usepackage{geometry}
\usepackage{makeidx}
\usepackage{graphicx}
\usepackage{caption}
\usepackage{verbatim}

\usepackage[colorlinks=true,allcolors=black]{hyperref}  % 链接相关, 设置为有颜色, 颜色是黑色, 下面选项类似
%\usepackage[colorlinks=true,linkcolor=black,citecolor=black,urlcolor=black]{hyperref}
%\usepackage[colorlinks=true]{hyperref}

\usepackage[nameinlink]{cleveref}  % 针对章节、图表和公式的**标签**
%\usepackage[labelsep=space,labelformat=dash]{caption}  % 用于设置图表和表格标题的格式, 分别是标签和文本有空格和标签和文本有破折号

\makeindex

% ctex
% \setCJKfamilyfont{BoldFont=SimHei,ItalicFont=KaiTi}{SimSun}  % 粗体使用SimHei, 斜体使用KaiTi
\setCJKmainfont[BoldFont=SimHei]{SimSun}  % 使用SimHei作为中文粗体  % 对应rm罗马
\setCJKsansfont{SimHei}  % 使用SimHei作为中文无衬线字体  % 对应sf无衬线
\setCJKmonofont{FangSong}  % 使用FangSong作为中文等宽字体  % 对应tt打字机
% fontspec, 是XeLaTeX和LuaLaTeX字体选择工具
\setmainfont{Times New Roman}  % 使用Times New Roman作为主要西文字体
\setmonofont{Times New Roman}  % 使用Times New Roman作为等宽西文字体


\geometry{left=3.18cm,right=3.18cm,top=2.54cm,bottom=2.54cm}  % 设置页边距
\linespread{1.5}  % 设置行距 (latex基础行距默认为1.2, 而word中默认为1.3)

\ctexset{
	chapter={
		name={第,章},
		number=\arabic{chapter},
		format=\raggedright\bfseries\zihao{3},
		beforeskip=0ex,
		afterskip=0.5ex,
		aftername=\enspace,
	},
	section={
		format=\raggedright\bfseries\zihao{4},
		beforeskip=0.5ex,
		afterskip=0ex,
		aftername=\enspace,
	},
	subsection={
		format=\raggedright\bfseries\zihao{-4},
		beforeskip=1ex,
		afterskip=1ex,
		aftername=\enspace,
	},
	contentsname={\hfill\zihao{3}\textbf{目 \quad 录}\hfill}
}

% \usepackage{titlesec}  % 设置标题样式 
% \titleformat{\chapter}{\zihao{3}\bfseries}{第\,\arabic{chapter}\,章}{\enspace}{}  % 设置一级标题左对齐,中文字体为黑体,英文字体为Times New Roman,字号为三号

% \DeclareCaptionLabelFormat{dash}{#1 \arabic{chapter}--\arabic{figure}}
% \DeclareCaptionLabelSeparator{dash}{-}
\renewcommand{\figurename}{图}
\renewcommand{\tablename}{表}
\crefname{figure}{图}{图}
\crefname{table}{表}{表}
\renewcommand{\thefigure}{\arabic{chapter}--\arabic{figure}}
\renewcommand{\thetable}{\arabic{chapter}--\arabic{table}}
% \makeatletter
% \crefformat{figure}{#2\cref@figure@name~\thefigure#3}
% \makeatother
% \captionsetup[figure]{labelsep=space,labelformat=dash,font={small,bf}}
\DeclareCaptionFont{five}{\zihao{5}\bfseries}
\captionsetup[figure]{labelsep=space,font=five}
\captionsetup[table]{labelsep=space,font=five,position=top}

\usepackage{titling}
% Title Page
\title{论文题目}
%\author{作者姓名}

\usepackage{fancyhdr}  % 定义页眉页脚

\fancypagestyle{tocstyle}{
	\fancyhf{} % clear all header and footer fields
	\renewcommand{\headrulewidth}{0pt}
	\fancyfoot[C]{\zihao{-5}\thepage}
}
\fancypagestyle{mainstyle}{
	\fancyhf{}
	\renewcommand{\headrulewidth}{0.3pt}
	\fancyhead[CO]{\zihao{5}\texttt{浙江理工大学本科毕业设计(论文)}}
	\fancyhead[CE]{\zihao{5}\texttt{\thetitle}}
	\fancyfoot[CE,CO]{\zihao{-5}\thepage}
}

\usepackage{placeins}
\usepackage{xcolor}
\usepackage[cache=false]{minted}
\setminted{
	frame=lines,
	framesep=2mm,
	baselinestretch=1.2,
%	bgcolor=lightgray,
	fontsize=\footnotesize,
	linenos
}

\graphicspath{{figure/}}

\usepackage{tocloft}
\setlength\cftbeforetoctitleskip{0ex}
\setlength\cftaftertoctitleskip{0ex}
\setlength\cftbeforechapskip{0ex}
\setlength\cftbeforepartskip{0ex}
\renewcommand{\cftpartfont}{\zihao{4}\bfseries}
\renewcommand{\cftsecfont}{\zihao{5}\mdseries}
\renewcommand{\cftsubsecfont}{\zihao{5}\ttfamily}
\renewcommand{\cftchappagefont}{\mdseries}
\renewcommand{\cftchapleader}{\cftdotfill{\cftdotsep}}

\usepackage{pdfpages}

% GB/T 7714-2015标准
\usepackage[backend=biber,style=gb7714-2015,gbnoauthor=true]{biblatex}
\addbibresource{thesis.bib}

\begin{document}
\zihao{-4}

\includepdf[pages=-]{诚信声明.pdf}
\phantomsection
\pdfbookmark[0]{摘要}{bookmark:摘要}
\cftaddtitleline{toc}{part}{摘 \quad 要}{}
\renewcommand{\abstractname}{\textbf{\zihao{-3}摘 \quad 要}}
\begin{abstract}

    这里是摘要。

    可以有多个自然段。

    一个自然段足够长会是什么效果呢呢呢呢呢呢呢呢呢呢呢呢呢呢呢呢呢呢呢呢呢呢呢呢呢呢呢呢呢呢呢呢呢呢呢呢呢呢呢呢呢呢呢呢呢呢呢呢呢呢?
	
	\vspace{0.5ex}
	\noindent\textbf{关键词:}关键词1;关键词2;关键词3;关键词4;关键词5
\end{abstract}
\clearpage


\phantomsection
\pdfbookmark[0]{Abstract}{bookmark:Abstract}
\cftaddtitleline{toc}{part}{Abstract}{}
\renewcommand{\abstractname}{\textbf{\zihao{-3}Abstract}}
\begin{abstract}
    % 这里的\qquad必须有, 否则英文摘要第一自然段不能正确缩进
	\qquad Here is the abstract.
    
    % 第二段则不需要
    There can be multiple natural segments.
    
    What is the effect of a natural segment being long enoughhhhhhhhhhhhhhhhhhhhhhhhhhhhhhhhhhhhhhhhhhhhhhhhhh?

	\vspace{0.5ex}
	\noindent\textbf{Keywords: }Keywords1; Keywords2; Keywords3; Keywords4; Keywords5
\end{abstract}
\clearpage


\pdfbookmark[0]{目录}{bookmark:目录}
\pagestyle{tocstyle}
\fancypagestyle{plain}{
    \pagestyle{tocstyle}
}
\pagenumbering{Roman}
\tableofcontents

\clearpage
\pagestyle{mainstyle}
\fancypagestyle{plain}{
    \pagestyle{mainstyle}
}
\pagenumbering{arabic}

\chapter{看待LaTeX语法的一个视角}

\begin{itemize}
    \item {
        语法教程链接
        \begin{itemize}
            \item \href{https://zhuanlan.zhihu.com/p/521649367?utm_source=zhihu}{知乎·槿灵兮·【LaTeX】针对萌新自学者的入门教程}
        \end{itemize}
    }
\end{itemize}

\section{如何实现内容和格式分离}


笔者仅在这里表达对LaTeX语法设计哲学的思考。如果读者之前接触过XML、HTML这样的标记语言,甚至接触过Typst这样LaTeX的“竞品”,更能对LaTeX的语法设计有感悟。

作为标记语言,源码仅是文本文件(对比Word的二进制文件),那么如何在文本文件中同时表达内容和格式呢?这就是LaTeX中的Command——以\verb|\|开头的命令。

\begin{itemize}
    \item 命令可以直接使用,比如\verb|\checkmark|是\checkmark,\verb|\makeindex|是让项目建立索引。
    \item 命令可以接参数,比如\verb|\subsection{子标题}|表示创建一个子标题,并用\verb|{}|中的内容作为子标题内容。多个参数由多个\verb|{}|包括。
    \item 命令可以接选项,比如\verb|\command[]{}|,这里的\verb|[]|中可以添加参数。
    \item 命令还有变体,比如\verb|\section{}|和\verb|\section*{}|效果是不一样的。
\end{itemize}

上面的后两者在论文中不用关心。

在LaTeX中可以用大括号\verb|{}|包裹内容,除此之外,还能用\verb|\begin|和\verb|\end|包裹内容,当然,这也是命令,可以接受参数,后面会讲到。

\section{标记系统中如何处理空白}

如果读者学过编译原理,就了解token这个概念,语法中划分通常使用空白划分token,比如空格、换行、制表。但是在标记系统中,内容和命令是共存的,编译器怎么知道空白是命令的部分还是属于内容本身?

\begin{itemize}
    \item 文本和文本之间,空格和制表是被忽略的,想添加空格需要使用命令\verb|\quad|;换行相当于空格,想换段需要空行。
    \item 文本和命令以及命令相关符号(大括号和中括号)之间,无参数的命令和之后的文本肯定是需要空白的,不然无法划分token,命令和之前的文本的空格是有意义的,因为有\verb|\|划分,其他同上。
    \item 命令和命令之间是忽略空白的。
\end{itemize}

\chapter{标题语法}

每个章节的tex文件都以\verb|\chapter|开头,其参数为当前章节的标题。

\begin{itemize}
    \item 子标题使用命令\verb|\section|
    \item 二级子标题使用命令\verb|\subsection|
    \item 三级子标题使用命令\verb|\subsubsection|
    \item 没有四级子标题
\end{itemize}

\section{子标题}

\subsection{二级子标题}

\subsubsection{三级子标题}

\chapter{其他常用语法}

\section{强调}

\textbf{粗体}
% \textit{斜体}不支持

数学模式中

$$
    \mathbf{Bold} \ and \ \mathit{Italic}
    % 数学模式中不支持中文
$$

\section{列表}

\subsection{无序列表}

\begin{itemize}
    \item 一个
    \item 又一个
    \item 还一个
\end{itemize}

\subsection{有序列表}

\begin{enumerate}
    \item 第一个
    \item 第二个
    \item 第三个
\end{enumerate}

\section{代码}

不支持

% \begin{lstlisting}[style=CStyle,title=Hello Word]
%     #include <iostream>

%     int main() {
%         std::cout << "Hello World" << std::endl;
%         return 0;
%     }
% \end{lstlisting}
\chapter{链接}

对被链接物的命名以及排序会由LaTeX自动完成。

\section{图片}

\begin{figure}[H]
    \centering
    \includegraphics[width=0.6\linewidth]{g}  % 文件名(没有后缀名)
    \caption{浙江理工大学校徽}  % 图片名
    \label{fig:g}  % 图片标记名,用于引用
\end{figure}

\cref{fig:g}

\autoref{fig:g}  % 英文,不用

\section{表格}

\begin{table}[htbp]
    \caption{表格名}  % 表格名
    \label{tab:t}  % 表格标记名,用于引用
    \zihao{-5}
    \centering
    \begin{tabular}{c c c c}
        \hline
        表头1   & 表头2   & 表头3   \\
        \hline
        行1元素1 & 行1元素2 & 行1元素3 \\
        行2元素1 & 行2元素2 & 行2元素3 \\
        行3元素1 & 行3元素2 & 行3元素3 \\
        \hline
    \end{tabular}
\end{table}

\cref{tab:t}

\autoref{tab:t}  % 英文,不用

\section{公式}

\begin{equation}
    \label{eq:m}
    This \quad is \quad a \quad mathematical \quad formula.
\end{equation}

\ref{eq:m}

\section{文献}

喏\cite{vaswani_attention_2023}


{
\ctexset{chapter/format=\centering\bfseries\zihao{3}}
\chapter*{参考文献}

\phantomsection
\pdfbookmark[0]{参考文献}{bookmark:参考文献}
\cftaddtitleline{toc}{chapter}{参考文献}{\thepage}
\printbibliography[heading=none]
\clearpage
}
{
\ctexset{chapter/format=\centering\bfseries\zihao{3}}
\chapter*{致 \quad 谢}

\phantomsection
\pdfbookmark[0]{致谢}{bookmark:致谢}
\cftaddtitleline{toc}{chapter}{致 \quad 谢}{\thepage}

致谢内容

\begin{flushright}
姓名

xxxx 年 xx 月 xx 日
\end{flushright}
}

\end{document}
