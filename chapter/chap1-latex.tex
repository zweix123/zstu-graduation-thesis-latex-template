\chapter{看待LaTeX语法的一个视角}

\begin{itemize}
    \item {
        语法教程链接
        \begin{itemize}
            \item \href{https://zhuanlan.zhihu.com/p/521649367?utm_source=zhihu}{知乎·槿灵兮·【LaTeX】针对萌新自学者的入门教程}
        \end{itemize}
    }
\end{itemize}

\section{如何实现内容和格式分离}


笔者仅在这里表达对LaTeX语法设计哲学的思考。如果读者之前接触过XML、HTML这样的标记语言,甚至接触过Typst这样LaTeX的“竞品”,更能对LaTeX的语法设计有感悟。

作为标记语言,源码仅是文本文件(对比Word的二进制文件),那么如何在文本文件中同时表达内容和格式呢?这就是LaTeX中的Command——以\verb|\|开头的命令。

\begin{itemize}
    \item 命令可以直接使用,比如\verb|\checkmark|是\checkmark,\verb|\makeindex|是让项目建立索引。
    \item 命令可以接参数,比如\verb|\subsection{子标题}|表示创建一个子标题,并用\verb|{}|中的内容作为子标题内容。多个参数由多个\verb|{}|包括。
    \item 命令可以接选项,比如\verb|\command[]{}|,这里的\verb|[]|中可以添加参数。
    \item 命令还有变体,比如\verb|\section{}|和\verb|\section*{}|效果是不一样的。
\end{itemize}

上面的后两者在论文中不用关心。

在LaTeX中可以用大括号\verb|{}|包裹内容,除此之外,还能用\verb|\begin|和\verb|\end|包裹内容,当然,这也是命令,可以接受参数,后面会讲到。

\section{标记系统中如何处理空白}

如果读者学过编译原理,就了解token这个概念,语法中划分通常使用空白划分token,比如空格、换行、制表。但是在标记系统中,内容和命令是共存的,编译器怎么知道空白是命令的部分还是属于内容本身?

\begin{itemize}
    \item 文本和文本之间,空格和制表是被忽略的,想添加空格需要使用命令\verb|\quad|;换行相当于空格,想换段需要空行。
    \item 文本和命令以及命令相关符号(大括号和中括号)之间,无参数的命令和之后的文本肯定是需要空白的,不然无法划分token,命令和之前的文本的空格是有意义的,因为有\verb|\|划分,其他同上。
    \item 命令和命令之间是忽略空白的。
\end{itemize}
